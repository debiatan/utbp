%%%%%%%%%%%%%%%%%%%%%%%%%%%%%%%%%%%%%%%%%%%%%%%%%%%%%%%%%%%%
%%  This Beamer template was created by Cameron Bracken.
%%  Anyone can freely use or modify it for any purpose
%%  without attribution.
%%
%%  Last Modified: January 9, 2009
%%

\documentclass[xcolor=x11names,compress]{beamer}

%% General document %%%%%%%%%%%%%%%%%%%%%%%%%%%%%%%%%%
\usepackage{graphicx}
\usepackage{tikz}
\usetikzlibrary{lindenmayersystems}
\usepackage[utf8]{inputenc}
\usepackage{verbatim}
\usepackage{fancyvrb}
\usepackage{pifont}
\usepackage{mathrsfs}
\usepackage{pmboxdraw}
\usepackage{minted}
%%%%%%%%%%%%%%%%%%%%%%%%%%%%%%%%%%%%%%%%%%%%%%%%%%%%%%
% Code snippets
\newminted{python}{fontsize=\scriptsize, 
		   framesep=3mm}

%% Beamer Layout %%%%%%%%%%%%%%%%%%%%%%%%%%%%%%%%%%
\useoutertheme[subsection=false,shadow]{miniframes}
\useinnertheme{default}
\usefonttheme{serif}
\usepackage{palatino}

\setbeamerfont{title like}{shape=\scshape}
\setbeamerfont{frametitle}{shape=\scshape}

\setbeamercolor*{lower separation line head}{bg=DeepSkyBlue4}
\setbeamercolor*{normal text}{fg=black,bg=white}
\setbeamercolor*{alerted text}{fg=red}
\setbeamercolor*{example text}{fg=black}
\setbeamercolor*{structure}{fg=black}

\setbeamercolor*{palette tertiary}{fg=black,bg=black!10}
\setbeamercolor*{palette quaternary}{fg=black,bg=black!10}

\renewcommand{\(}{\begin{columns}}
\renewcommand{\)}{\end{columns}}
\newcommand{\<}[1]{\begin{column}{#1}}
\renewcommand{\>}{\end{column}}
%%%%%%%%%%%%%%%%%%%%%%%%%%%%%%%%%%%%%%%%%%%%%%%%%%
\setbeamertemplate{caption}[numbered]       % Numbered figures
\setbeamertemplate{navigation symbols}{}    % No navigation footer
%\setbeamertemplate{footline}[page number]

\begin{document}

% Code snippets
\defverbatim[colored]\codeFactorial{
\begin{pythoncode}
def factorial(n):
    """
    Returns the factorial of n
    (int -> int)

    >>> factorial(2)
    2
    >>> factorial(3)
    6
    """
    if n:
        return n*factorial(n-1)
    else:
        return 1
\end{pythoncode}
}

\defverbatim[colored]\codeFactorialTarget{
\begin{pythoncode*}{fontsize=\huge}
@UTBP
def factorial(n):
    """
    factorial(0) == 1
    factorial(1) == 1
    factorial(2) == 2
    factorial(3) == 6
    """
\end{pythoncode*}
}

\defverbatim[colored]\codeLengthIteration{
\begin{pythoncode*}{fontsize=\large}
def length(l):
    i = 0
    for e in l:
        i += 1
    return i
\end{pythoncode*}
}

\defverbatim[colored]\codeLengthRecursion{
\begin{pythoncode*}{fontsize=\large}
def length(l):
    if l:
        return 1+length(l[1:])
    else:
        return 0
\end{pythoncode*}
}

\defverbatim[colored]\codeLengthGoToHell{
\begin{pythoncode*}{fontsize=\large}
length = lambda l: sum(1 for e in l)
\end{pythoncode*}
}

\defverbatim[colored]\codeLengthRecursionSmall{
\begin{pythoncode*}{fontsize=\scriptsize}
def length(l):
    if l: return 1+length(l[1:])
    else: return 0
\end{pythoncode*}
}

\defverbatim[colored]\codeLengthExampleA{
\begin{pythoncode*}{fontsize=\large}
@UTBP
def length(l):
"""
length(()) == 0
"""
\end{pythoncode*}
}

\defverbatim[colored]\codeLengthExampleB{
\begin{pythoncode*}{fontsize=\large}
@UTBP
def length(l):
"""
length(()) == 0
length((2,)) == 1
"""
\end{pythoncode*}
}

\defverbatim[colored]\codeLengthExampleC{
\begin{pythoncode*}{fontsize=\Large}
@UTBP
def length(l):
"""
length(()) == 0
length((2,)) == 1
length((2, 3)) == 2
"""
\end{pythoncode*}
}

\defverbatim[colored]\codeImpossible{
\begin{pythoncode*}{fontsize=\scriptsize}
    @UTBP
    def imposible(l):
    """
    imposible(0) == 0
    imposible(0) == 1
    """
\end{pythoncode*}
}


%%%%%%%%%%%%%%%%%%%%%%%%%%%%%%%%%%%%%%%%%%%%%%%%%%%%%%
%%%%%%%%%%%%%%%%%%%%%%%%%%%%%%%%%%%%%%%%%%%%%%%%%%%%%%
\begin{frame}
    \title{Programación basada en pruebas unitarias}
\author{Miguel Lechón} 
\date{28 de diciembre de 2014}
\titlepage
\end{frame}

%%%%%%%%%%%%%%%%%%%%%%%%%%%%%%%%%%%%%%%%%%%%%%%%%%%%%%
%%%%%%%%%%%%%%%%%%%%%%%%%%%%%%%%%%%%%%%%%%%%%%%%%%%%%%
%\section*{\scshape Outline}
%\begin{frame}{Outline of the presentation}
%\tableofcontents[hideallsubsections]
%\end{frame}

%%%%%%%%%%%%%%%%%%%%%%%%%%%%%%%%%%%%%%%%%%%%%%%%%%%%%%
%%%%%%%%%%%%%%%%%%%%%%%%%%%%%%%%%%%%%%%%%%%%%%%%%%%%%%
\section{\scshape Introducción}
\subsection{SIGBOVIK}
\begin{frame}{Contexto -- SigBOVIK}
    \begin{figure}[t]
        \centering
        \includegraphics[width=1.0\textwidth]{images/sigbovik.png}
    \end{figure}
\end{frame}

%%%%%%%%%%%%%%%%%%%%%%%%%%%%%%%%%%%%%%%%%%%%%%%%%%%%%%
%%%%%%%%%%%%%%%%%%%%%%%%%%%%%%%%%%%%%%%%%%%%%%%%%%%%%%
\subsection{Yo}
\begin{frame}{Yo}
    \frametitle{Contexto -- Yo}
    \begin{itemize}
        \item Informático \pause
        \item Programador \textcolor{red}{sin código en producción} \pause
        \item Administrador de sistemas \textcolor{red}{mediocre} \pause
        \item Ex-estudiante de doctorado \textcolor{red}{no doctorado} \pause
        \item Actualmente, \textcolor{blue}{ama de casa}
    \end{itemize}
\end{frame}

%%%%%%%%%%%%%%%%%%%%%%%%%%%%%%%%%%%%%%%%%%%%%%%%%%%%%%
%%%%%%%%%%%%%%%%%%%%%%%%%%%%%%%%%%%%%%%%%%%%%%%%%%%%%%
\subsection{Motivación}
\begin{frame}{Motivación}
    ¿En qué consiste programar?
    \begin{columns}
        \begin{column}{.5\linewidth}
        \begin{block}
            \codeFactorial
        \end{block}
        \end{column}
        \begin{column}{.5\linewidth}
            \begin{itemize}\itemsep25pt \pause
                \item Explicar qué se espera del código\pause
                \item Explicarlo otra vez, por si las moscas\pause
                \item Explicarlo de nuevo, esta vez para el ordenador\pause
            \end{itemize}
        \end{column}
    \end{columns}
    Nadie aprende a programar para trabajar más de la cuenta. Tiene que haber una manera mejor\ldots
\end{frame}

%%%%%%%%%%%%%%%%%%%%%%%%%%%%%%%%%%%%%%%%%%%%%%%%%%%%%%
%%%%%%%%%%%%%%%%%%%%%%%%%%%%%%%%%%%%%%%%%%%%%%%%%%%%%%
\subsection{Programación basada en pruebas unitarias}
\begin{frame}{Programación basada en pruebas unitarias}
    \codeFactorialTarget \pause
    Sí, funciona.
\end{frame}


%%%%%%%%%%%%%%%%%%%%%%%%%%%%%%%%%%%%%%%%%%%%%%%%%%%%%%
%%%%%%%%%%%%%%%%%%%%%%%%%%%%%%%%%%%%%%%%%%%%%%%%%%%%%%
\subsection{Pero\ldots ¿cómo?}
\begin{frame}{Pero\ldots ¿cómo?}
    Premisas:
    \pause
    \begin{itemize}
        \item Toda función se puede escribir de forma recursiva \pause
        \item Toda función recursiva puede representarse mediante un árbol \pause
        \item Los árboles se pueden enumerar en orden creciente de complejidad \pause
    \end{itemize}
    Conclusión:

    \textbf{$\rightarrow$ Las funciones se pueden enumerar en orden creciente de complejidad} \pause

    (y podemos comprobar, una a una, si satisfacen un conjunto arbitrario de pruebas unitarias)
\end{frame}

%%%%%%%%%%%%%%%%%%%%%%%%%%%%%%%%%%%%%%%%%%%%%%%%%%%%%%
%%%%%%%%%%%%%%%%%%%%%%%%%%%%%%%%%%%%%%%%%%%%%%%%%%%%%%

\section{\scshape Premisas}

\subsection{Toda función se puede escribir de forma recursiva}
\begin{frame}[fragile]
    \frametitle{Recursividad para todo}
    \textcolor{red}{Advertencia:} Este es un ejercicio académico \pause extremadamente complejo\pause. Gritad si os perdéis. \pause
    \begin{columns}
        \begin{column}{.30\linewidth}
        \begin{block}
            \codeLengthIteration
        \end{block}
        \end{column} \pause
        \begin{column}{.65\linewidth}
        \begin{block}
            \codeLengthRecursion
        \end{block}
        \end{column}
    \end{columns} \pause
    La versión recursiva es claramente superior.

    \pause\codeLengthGoToHell

\end{frame}

%%%%%%%%%%%%%%%%%%%%%%%%%%%%%%%%%%%%%%%%%%%%%%%%%%%%%%
%%%%%%%%%%%%%%%%%%%%%%%%%%%%%%%%%%%%%%%%%%%%%%%%%%%%%%

\subsection{Expresiones como árboles}
\begin{frame}[fragile]
    \frametitle{Expresiones como árboles}
    $$\frac{-b\pm\sqrt{b^2-4ac}}{2a}$$
    
    \pause

\begin{Verbatim}[commandchars=\\\{\},codes={\catcode`$=3\catcode`_=8}]
        \textbf{/}
        ├── \textbf{+/-}
        │   ├── \textbf{neg}
        │   │   └── b
        │   └── \textbf{sqrt}
        │       └── \textbf{-}
        │           ├── \textbf{^}
        │           │   ├── b
        │           │   └── 2
        │           └── \textbf{*}
        │               ├── 4
        └── \textbf{*}           └── \textbf{*}
            ├── 2           ├── a
            └── a           └── c
\end{Verbatim}
\end{frame}



\subsection{Toda función recursiva puede representarse mediante un árbol}
\begin{frame}[fragile]
    \frametitle{Funciones recursivas como árboles}
    \codeLengthRecursionSmall
\begin{Verbatim}[commandchars=\\\{\},codes={\catcode`$=3\catcode`_=8}]
\textbf{python}                \textbf{pseudo-scheme}
def                   define                    
  length              ├── length                
                      └── lambda                
(l):                      ├── l                 
    if                    └── if                
      l:                      ├── l             
        return 1+             ├── succ          
            length(           │   └── length    
                l[1:]         │       └── cdr   
            )                 │           └── l 
    else: return 0            └── 0             
\end{Verbatim}
\end{frame}

\subsection{Los árboles se pueden enumerar en orden creciente de complejidad \pause}
\begin{frame}[fragile]
    \frametitle{Enumeración de árboles}
\begin{Verbatim}[commandchars=\\\{\},codes={\catcode`$=3\catcode`_=8}]
     [\textbf{1}]   [  \textbf{2}  ]    [        \textbf{3}       ]
      r     r          r         r
            └── l      └── b     ├── l0
                           └── l └── l1
\end{Verbatim}

\pause

\begin{Verbatim}[commandchars=\\\{\},codes={\catcode`$=3\catcode`_=8}]
     [               \textbf{4}                 ]
      r          r          r         
      └── b0     └── b0     ├── b0    
          └── b1     ├── l0 │   └── l0
              └── l0 └── l1 └── l1    
            r             r
            ├── b0        ├── l0
            └── b1        ├── l1
                └── l0    └── l2
\end{Verbatim}



\end{frame}

\section{\scshape Ingredientes}
\subsection{Ingredientes}
\begin{frame}{Ingredientes del paradigma UTBP}
    \begin{itemize}
        \item Estructuras arborescentes \pause \emph{(déjà vu)} \pause 
        \item Pobladas por operaciones: \pause
            \begin{itemize}
                \item Funciones aritméticas (\texttt{succ}, \texttt{pred}) \pause
                \item Manipulación de listas (\texttt{car}, \texttt{cdr}, \texttt{cons}, \texttt{list?}) \pause
                \item Formas especiales (\texttt{if}) \pause
            \end{itemize}
        \item Sobre dos tipos: \pause
            \begin{itemize}
            \item Números naturales no negativos \pause
            \item Listas, posiblemente anidadas (realmente tuplas, porque no pensamos mutarlas)\pause
            \end{itemize}
        \item Constantes (operaciones sobre cero elementos): \pause
            \begin{itemize}
                \item \texttt{0} \pause
                \item \texttt{()}
            \end{itemize}
    \end{itemize}
\end{frame}

\section{\scshape Ejemplo de uso}
\subsection{Ejemplo}
\begin{frame}[fragile]
    \frametitle{Ejemplo de uso -- Longitud de una lista}\pause
    \begin{columns}
        \begin{column}{.50\linewidth}
        \begin{block}
            \codeLengthExampleA
        \end{block}
        \end{column}\pause
        \begin{column}{.50\linewidth}
        \begin{verbatim}
define
├── length
└── lambda
    ├── l
    └── 0
        \end{verbatim}
        \end{column}
    \end{columns}\pause

    \begin{columns}
        \begin{column}{.50\linewidth}
        \begin{block}
            \codeLengthExampleB
        \end{block}
        \end{column}\pause
        \begin{column}{.50\linewidth}
        \begin{verbatim}
define
├── length
└── lambda
    ├── l
    └── if
        ├── l
        ├── list?
        │   └── l
        └── 0
        \end{verbatim}
        \end{column}
    \end{columns}
\end{frame}

%%%%%%%%%%%%%%%%%%%%%%%%%%%%%%%%%%%%%%%%%%%%%%%%%%%%%%
%%%%%%%%%%%%%%%%%%%%%%%%%%%%%%%%%%%%%%%%%%%%%%%%%%%%%%

\subsection{Ejemplo}
\begin{frame}[fragile]
    \frametitle{Ejemplo de uso -- Longitud de una lista}
    \begin{columns}
        \begin{column}{.50\linewidth}
        \begin{block}
            \codeLengthExampleC
        \end{block}
        \end{column}\pause
        \begin{column}{.50\linewidth}
        \begin{verbatim}
define
├── length
└── lambda
    ├── l
    └── if
        ├── l
        ├── succ
        │   └── length
        │       └── cdr
        │           └── l
        └── 0
        \end{verbatim}
        \end{column}
    \end{columns} \pause

    Glorioso.
\end{frame}

%%%%%%%%%%%%%%%%%%%%%%%%%%%%%%%%%%%%%%%%%%%%%%%%%%%%%%
%%%%%%%%%%%%%%%%%%%%%%%%%%%%%%%%%%%%%%%%%%%%%%%%%%%%%%
\section{\scshape En conclusión}
\subsection{Características}
\begin{frame}{Características}
    \begin{itemize}
        \item Corrección garantizada \pause
            \begin{block}
                \codeImpossible
            \end{block} \pause
        \item Prevención de días del juicio final del estilo \emph{Terminator}
            (amurallado dentro de un intérprete de Scheme derivado de \texttt{http://norvig.com/lispy.html})\pause
        \item Interoperable con Python estándar \pause
        \item Complejidad de Kolmogorov mínima \pause
        \item Navajitud de Occam máxima
    \end{itemize}
\end{frame}

\subsection{Conclusión}
\begin{frame}{Conclusión}
    La programación imperativa es como muy del siglo XX. \pause

    El mundo necesita herramientas más simples y estúpidas 
    para programadores descuidados.\pause

    \vspace{1in}

    {\Large ¿Preguntas?}
\end{frame}


\end{document}
