%%%%%%%%%%%%%%%%%%%%%%%%%%%%%%%%%%%%%%%%%%%%%%%%%%%%%%%%%%%%
%%  This Beamer template was created by Cameron Bracken.
%%  Anyone can freely use or modify it for any purpose
%%  without attribution.
%%
%%  Last Modified: January 9, 2009
%%

\documentclass[xcolor=x11names,compress]{beamer}

%% General document %%%%%%%%%%%%%%%%%%%%%%%%%%%%%%%%%%
\usepackage{graphicx}
\usepackage{tikz}
\usetikzlibrary{lindenmayersystems}
\usepackage[utf8]{inputenc}
\usepackage{verbatim}
\usepackage{pifont}
\usepackage{mathrsfs}
\usepackage{minted}
%%%%%%%%%%%%%%%%%%%%%%%%%%%%%%%%%%%%%%%%%%%%%%%%%%%%%%
% Code snippets
\newminted{python}{fontsize=\scriptsize, 
		   framesep=3mm}

%% Beamer Layout %%%%%%%%%%%%%%%%%%%%%%%%%%%%%%%%%%
\useoutertheme[subsection=false,shadow]{miniframes}
\useinnertheme{default}
\usefonttheme{serif}
\usepackage{palatino}

\setbeamerfont{title like}{shape=\scshape}
\setbeamerfont{frametitle}{shape=\scshape}

\setbeamercolor*{lower separation line head}{bg=DeepSkyBlue4}
\setbeamercolor*{normal text}{fg=black,bg=white}
\setbeamercolor*{alerted text}{fg=red}
\setbeamercolor*{example text}{fg=black}
\setbeamercolor*{structure}{fg=black}

\setbeamercolor*{palette tertiary}{fg=black,bg=black!10}
\setbeamercolor*{palette quaternary}{fg=black,bg=black!10}

\renewcommand{\(}{\begin{columns}}
\renewcommand{\)}{\end{columns}}
\newcommand{\<}[1]{\begin{column}{#1}}
\renewcommand{\>}{\end{column}}
%%%%%%%%%%%%%%%%%%%%%%%%%%%%%%%%%%%%%%%%%%%%%%%%%%
\setbeamertemplate{caption}[numbered]       % Numbered figures
\setbeamertemplate{navigation symbols}{}    % No navigation footer
%\setbeamertemplate{footline}[page number]

\begin{document}

% Code snippets
\defverbatim[colored]\codeFactorial{
\begin{pythoncode}

def factorial(n):
    """
    Returns the factorial of n
    (int -> int)

    >>> factorial(2)
    2
    >>> factorial(3)
    6
    """
    if n:
        return n*factorial(n-1)
    else:
        return 1
\end{pythoncode}
}

\defverbatim[colored]\codeFactorialTarget{
\begin{pythoncode*}{fontsize=\huge}
@UTBP
def factorial(n):
    """
    factorial(0) == 1
    factorial(1) == 1
    factorial(2) == 2
    factorial(3) == 6
    """
\end{pythoncode*}
}


%%%%%%%%%%%%%%%%%%%%%%%%%%%%%%%%%%%%%%%%%%%%%%%%%%%%%%
%%%%%%%%%%%%%%%%%%%%%%%%%%%%%%%%%%%%%%%%%%%%%%%%%%%%%%
\begin{frame}
    \title{Programación basada en pruebas unitarias}
\author{Miguel Lechón} 
\date{28 de diciembre de 2014}
\titlepage
\end{frame}

%%%%%%%%%%%%%%%%%%%%%%%%%%%%%%%%%%%%%%%%%%%%%%%%%%%%%%
%%%%%%%%%%%%%%%%%%%%%%%%%%%%%%%%%%%%%%%%%%%%%%%%%%%%%%
%\section*{\scshape Outline}
%\begin{frame}{Outline of the presentation}
%\tableofcontents[hideallsubsections]
%\end{frame}

%%%%%%%%%%%%%%%%%%%%%%%%%%%%%%%%%%%%%%%%%%%%%%%%%%%%%%
%%%%%%%%%%%%%%%%%%%%%%%%%%%%%%%%%%%%%%%%%%%%%%%%%%%%%%
\section{\scshape Introducción}
\subsection{SIGBOVIK}
\begin{frame}{Contexto -- SigBOVIK}
    \begin{figure}[t]
        \centering
        \includegraphics[width=1.0\textwidth]{images/sigbovik.png}
    \end{figure}
\end{frame}

%%%%%%%%%%%%%%%%%%%%%%%%%%%%%%%%%%%%%%%%%%%%%%%%%%%%%%
%%%%%%%%%%%%%%%%%%%%%%%%%%%%%%%%%%%%%%%%%%%%%%%%%%%%%%
\subsection{Yo}
\begin{frame}{Yo}
    \frametitle{Contexto -- Yo}
    \begin{itemize}
        \item Informático \pause
        \item Programador \textcolor{red}{sin código en producción} \pause
        \item Administrador de sistemas \textcolor{red}{mediocre} \pause
        \item Ex-estudiante de doctorado \textcolor{red}{sin doctorado} \pause
        \item Actualmente, \textcolor{blue}{ama de casa}
    \end{itemize}
\end{frame}

%%%%%%%%%%%%%%%%%%%%%%%%%%%%%%%%%%%%%%%%%%%%%%%%%%%%%%
%%%%%%%%%%%%%%%%%%%%%%%%%%%%%%%%%%%%%%%%%%%%%%%%%%%%%%
\subsection{Motivación}
\begin{frame}{Motivación}
    ¿En qué consiste programar?
    \begin{columns}
        \begin{column}{.5\linewidth}
        \begin{block}
            \codeFactorial
        \end{block}
        \end{column}
        \begin{column}{.5\linewidth}
            \begin{itemize}\itemsep25pt \pause
                \item Explicar qué se espera del código\pause
                \item Explicarlo otra vez, por si las moscas\pause
                \item Explicarlo de nuevo, esta vez para el ordenador\pause
            \end{itemize}
        \end{column}
    \end{columns}
    Uno no aprende a programar para trabajar más de la cuenta.
\end{frame}

%%%%%%%%%%%%%%%%%%%%%%%%%%%%%%%%%%%%%%%%%%%%%%%%%%%%%%
%%%%%%%%%%%%%%%%%%%%%%%%%%%%%%%%%%%%%%%%%%%%%%%%%%%%%%
\subsection{Objetivo}
\begin{frame}{Objetivo}
    \pause
    \codeFactorialTarget
\end{frame}


%%%%%%%%%%%%%%%%%%%%%%%%%%%%%%%%%%%%%%%%%%%%%%%%%%%%%%
%%%%%%%%%%%%%%%%%%%%%%%%%%%%%%%%%%%%%%%%%%%%%%%%%%%%%%
\subsection{But Why?}
\begin{frame}{But Why?}
    \begin{itemize}
        \item I thought some of the criticism might graze our research
        \item What is the big deal with $\frac{1}{f}$ noise and power-law distributions and systems at the edge of criticality? 
    \end{itemize}
\end{frame}

%%%%%%%%%%%%%%%%%%%%%%%%%%%%%%%%%%%%%%%%%%%%%%%%%%%%%%
%%%%%%%%%%%%%%%%%%%%%%%%%%%%%%%%%%%%%%%%%%%%%%%%%%%%%%

\section{\scshape The original articles}

\subsection{Choosing words}
\begin{frame}{Choosing words}

\begin{itemize}
    \item Promises explanation; does not deliver
    \item \emph{Speech is not modular, it's more complicated}
    \item Favorite sentence:
\end{itemize}

\end{frame}

%%%%%%%%%%%%%%%%%%%%%%%%%%%%%%%%%%%%%%%%%%%%%%%%%%%%%%
%%%%%%%%%%%%%%%%%%%%%%%%%%%%%%%%%%%%%%%%%%%%%%%%%%%%%%

\subsection{Ultrafast action and cognition}
\begin{frame}{Ultrafast action and cognition}


\begin{itemize}
    \item \emph{Neuronal control cannot explain ultrafast reaction}
    \item \emph{Scale-freedom is everywhere}
    \item \emph{This thing hast to be complex}
    \item Maybe it is just a review, but it sounds like the Rumpelstiltskin approach to neuroscience (\emph{neuromusculoskeletal synergies are the solution!})
\end{itemize}
\end{frame}

\subsection{Brain-body-niche}
\begin{frame}{Brain-body-niche}

\begin{itemize}
    \item \emph{No need to dissociate brain from body or body from environment}
    \item \emph{Who needs internal representations when you're part of the system you're trying to represent?}
    \item \emph{No computation}
\end{itemize}
\end{frame}

\subsection{Multifractal fractality}
\begin{frame}{Multifractal fractality}
\begin{itemize}
    \item \emph{Power-laws imply complexity. Exponents, exponents!}
    \item \emph{Forget about computation. Let's talk about energy flow.}
    \item \emph{No computation}
\end{itemize}

\end{frame}

\section{\scshape Criticism}
\subsection{Criticism}
\begin{frame}{Criticism}
    Approach taken by three of the articles
    \begin{itemize}
        \item \emph{Detect 1/f noise in time series of human performance}
            \begin{itemize}
                \item Competing models?
                \item Frailty of the measure
            \end{itemize}
        \item \emph{The presence of 1/f noise implies an interaction dominant system}
            \begin{itemize}
                \item Other systems can also generate 1/f noise
                \item Generation of 1/f noise in complex systems depends on specific circumstances
            \end{itemize}
        \item \emph{Cognition is interaction-dominant and traditional approaches should be abandoned}
            \begin{itemize}
                \item The complex systems approach should explain something that traditional approaches cannot explain.
                \item Model instead of ``solving'' by dropping terms like self-organized criticality
            \end{itemize}
    \end{itemize}
\end{frame}

\subsection{Piles of rice and salt}
\begin{frame}{Piles of rice and salt}
\end{frame}

\subsection{Does this apply to us?}
\begin{frame}{Does this apply to us?}
    I don't think much of the discussion applies to us (the original Frontiers article should not have confronted Wagenmakers and Friston and we \emph{do} talk about components as well as their interactions), however\ldots
    \begin{itemize}
        \item Do we make our best effort to disprove criticality? Competing hypotheses.
        \item Do we try to model something other than 1/f?
    \end{itemize}
\end{frame}

%%%%%%%%%%%%%%%%%%%%%%%%%%%%%%%%%%%%%%%%%%%%%%%%%%%%%%
%%%%%%%%%%%%%%%%%%%%%%%%%%%%%%%%%%%%%%%%%%%%%%%%%%%%%%

\section{\scshape Doubts}
\subsection{Doubts}
\begin{frame}{Doubts}
I'm no statistical physicist and I should not take on blind faith some of 
    \begin{itemize}
        \item Power-laws and fractals
        \item Power-laws and complexity
    \end{itemize}
\end{frame}

\end{document}
